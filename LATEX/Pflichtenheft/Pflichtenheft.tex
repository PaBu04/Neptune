%% Based on techreport.tex template as sent by Erik Burger on 2023-11-20
%% 
%% Karlsruhe Institute of Technology
%% Institute for Program Structures and Data Organization
%% Chair for Software Design and Quality (SDQ)
%%
%% Dr.-Ing. Erik Burger
%% burger@kit.edu
%%
%% See https://sdq.kastel.kit.edu/wiki/Dokumentvorlagen
%%
%% Version 1.0, 2023-11-20

%% Available page modes: oneside, twoside
%% Available languages: english, ngerman
%% Available modes: draft, final (see README)
\documentclass[oneside, ngerman]{sdqtechreport}

%% ---------------------------------
%% | Information about the document |
%% ---------------------------------

%% Name of the group and authors
\author{von Neptun - \\
Paul Buda, Martin Scheuermann, Stephan Schneider, \\
Simon Schütz und Nils Seibert}

%% Title (and possibly subtitle) of the thesis
\title{Pflichtenheft}

\subtitle{zur Android-App Neptune}

%% You can put a logo in the ``logos'' directory and include it here
%% instead of the SDQ logo
% \grouplogo{myfile}
%% Alternatively, you can disable the group logo
% \nogrouplogo

\date{01.12.2023}

%% For example texts -- please remove in the final version
\usepackage{blindtext}

%% ====================================
%% ====================================
%% ||                                ||
%% || Beginning of the main document ||
%% ||                                ||
%% ====================================
%% ====================================
\begin{document}

%% Set PDF metadata
\setpdf

%% Set the title
\maketitle

%% ------------------------
%% |   Table of Contents  |
%% ------------------------
\tableofcontents

%% -----------------
%% |   Main part   |
%% -----------------
\cleardoublepage

%% -------------------
%% | Example content |
%% -------------------

\chapter{Einleitung}
\label{chap:Einleitung}



\chapter{Zielbestimmungen}
\label{chap:Zielbestimmungen}

\section{Musskriterien}
\label{sec:Zielbestimmungen:Musskriterien}

\section{Wunschkriterien}
\label{sec:Zielbestimmungen:Wunschkriterien}

\section{Abgrenzungskriterien}
\label{sec:Zielbestimmungen:Abgrenzungskriterien}



\chapter{Produkteinsatz}
\label{chap:Produkteinsatz}

\section{Anwendungsbereich}
\label{sec:Produkteinsatz:Anwendungsbereich}

\section{Zielgruppe}
\label{sec:Produkteinsatz:Zielgruppe}

\section{Betriebsbedingungen}
\label{sec:Produkteinsatz:Betriebsbedingungen}



\chapter{Produktumgebung}
\label{chap:Produktumgebung}

\section{Software}
\label{sec:Produktumgebung:Software}

\section{Hardware}
\label{sec:Produktumgebung:Hardware}



\chapter{Produktfunktionen}
\label{chap:Produktfunktionen}

\section{Grundfunktionen}
\label{sec:Produktfunktionen:Software}

\section{Erweiterte Funktionen}
\label{sec:Produktfunktionen:Hardware}



\chapter{Produktdaten}
\label{chap:Produktdaten}

\section{Systemdaten}
\label{sec:Produktdaten:Systemdaten}

\section{Benutzerdaten}
\label{sec:Produktdaten:Benutzerdaten}



\chapter{Systemmodell}
\label{chap:Systemmodell}



\chapter{Produktleistungen}
\label{chap:Produktleistungen}



\chapter{Benutzeroberfläche}
\label{chap:Benutzeroberfläche}

\section{Einführung}
\label{sec:Benutzeroberfläche:Einführung}

\section{Erklärungen}
\label{sec:Benutzeroberfläche:Erklärungen}

\section{Grafikentwürfe}
\label{sec:Benutzeroberfläche:Grafikentwürfe}



\chapter{Qualitätszielbestimmungen}
\label{chap:Qualitätszielbestimmungen}



\chapter{Testfälle und -szenarien}
\label{chap:Tests}

\section{Testfälle}
\label{sec:Tests:Testfälle}

\section{Testszenarien}
\label{sec:Tests:Testszenarien}



\chapter{Entwicklungsumgebung}
\label{chap:Entwicklungsumgebung}

\section{Software}
\label{sec:Entwicklungsumgebung:Software}

\section{Hardware}
\label{sec:Entwicklungsumgebung:Hardware}



\chapter{Begriffserklärungen}
\label{chap:Begriffserklärungen}


\section{}
\end{document}
