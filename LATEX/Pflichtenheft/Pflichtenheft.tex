%% Based on techreport.tex template as sent by Erik Burger on 2023-11-20
%% 
%% Karlsruhe Institute of Technology
%% Institute for Program Structures and Data Organization
%% Chair for Software Design and Quality (SDQ)
%%
%% Dr.-Ing. Erik Burger
%% burger@kit.edu
%%
%% S ee https://sdq.kastel.kit.edu/wiki/Dokumentvorlagen
%%
%% Version 1.0, 2023-11-20

%% Available page modes: oneside, twoside
%% Available languages: english, ngerman
%% Available modes: draft, final (see README)
\documentclass[oneside, ngerman]{sdqtechreport}

%% ---------------------------------
%% | Information about the document |
%% ---------------------------------

%% Name of the group and authors
\author{von Neptun - \\
Paul Buda, Martin Scheuermann, Stephan Schneider, \\
Simon Schütz und Nils Seibert}

%% Title (and possibly subtitle) of the thesis
\title{Pflichtenheft}

\subtitle{zur Android-App Neptune}

%% You can put a logo in the ``logos'' directory and include it here
%% instead of the SDQ logo
% \grouplogo{myfile}
%% Alternatively, you can disable the group logo
% \nogrouplogo

\date{01.12.2023}

%% For example texts -- please remove in the final version
\usepackage{blindtext}

%% ====================================
%% ====================================
%% ||                                ||
%% || Beginning of the main document ||
%% ||                                ||
%% ====================================
%% ====================================
\begin{document}

%% Set PDF metadata
\setpdf

%% Set the title
\maketitle

%% ------------------------
%% |   Table of Contents  |
%% ------------------------
\tableofcontents

%% -----------------
%% |   Main part   |
%% -----------------
\cleardoublepage

%% -------------------
%% | Example content |
%% -------------------

\chapter{Einleitung}
\label{chap:Einleitung}

\section{Einführung}
\label{sec:Einleitung:Einführung}
Musik spielt im Leben vieler Menschen eine enorm wichtige Rolle. Insbesondere bei Partys und Zusammenkünften mit Freunden sorgt eine gute Musikauswahl für eine gute Stimmung unter den Anwesenden. Aber auch der umgekehrte Fall ist vielen sicherlich gut bekannt – gefällt die abgespielte Musik den Anwesenden nicht, so kann dies die Stimmung erheblich trüben.

Nahezu alle der heute gängigen Musikstreaming-Anbieter versuchen bereits mit proprietären Lösungen, dem entgegenzuwirken. In der Praxis jedoch sind die eigens von den Anbietern angebotenen Lösungen häufig nicht praktikabel. Die Ursachen hierfür sind divers, so setzen die von den Anbietern selbst entwickelten Tools häufig voraus, dass alle Teilnehmer über ein bestehendes Abonnement beim entsprechenden Anbieter verfügen. 

Das Ziel des Projekts „Neptune“ ist es, eine praktikable Lösung für die eingangs beschriebene Problematik anzugeben. Hierzu soll im Rahmen des Moduls „Praxis der Softwareentwicklung“ eine Android-App entwickelt werden, mithilfe derer über die bei einer Party oder einem vergleichbaren Event abgespielte Musik entschieden werden kann.

Hierzu sollen die Anwesenden in verschiedenen verfügbaren Abstimmungs-Modi Musikvorschläge einbringen und über diese abstimmen können. Die Musik soll dann über das Endgerät einer weiteren anwesenden Person, des sogenannten „Hosts“, abgespielt werden. 
Mittels der Einbindung eines gängigen Musikstreaming-Service soll die App in die Lage versetzt werden, einen breiten Musikkatalog bereitzustellen.


\section{Anwendungsbereich}
\label{sec:Einleitung:Anwendungsbereich}

Das Ziel von Neptune ist es, die Musikauswahl bei privaten Veranstaltungen, wie zum Beispiel studentische WG- und Wohnheimpartys, einfacher und gerechter zu gestalten.
Die Android-App Neptune bietet Gruppen die Möglichkeit jeden an der Musikauswahl zu beteiligen und über die Abspielreihenfolge demokratisch abzustimmen. Durch die Integration mit einem Audio-Streaming-Dienst wie Spotify kann Neptune automatisch die am besten bewerteten Songs abspielen. Damit ist es möglich Songwünsche von Gästen zu erfüllen, ohne einen aktiven "DJ"  der die Wünsche entgegen nimmt und sie manuell in die Warteschlange hinzufügt.

Bei der Erstellung einer Listening-Session kann der Gastgeber die verfügbaren Lieder nach Belieben auf verschiedene Genres oder eine Playlist beschränken. Nach Auswahl des Modus kann der Gastgeber die Gäste einladen, sich an der Musikauswahl zu beteiligen, indem er einen sechsstelligen Zahlencode oder einen Link weitergibt. Die Gäste können dann in der App Lieder in die Vorschlagsliste hinzufügen und mit dem Verteilen von Likes Lieder in der Liste nach oben voten und sie somit schneller zum Abspielen bringen. Der Host hat durch eine Kontrollansicht einen Überblick über die Queue, sowie die Vorschlagsliste in Neptune. Er kann in dieser Ansicht Songs in die Queue hinzufügen und entfernen, sowie Lieder aus der Vorschlagsliste entfernen, dadurch hat der Host weiterhin die volle Kontrolle über die Musikauswahl. Zusätzlich kann er durch die Einstellung
eines Cooldowns verhindern das diesselben Songs mehrmals mit geringen Abstand gespielt werden.

Eine vollständige Beschreibung der Anwendungsfälle sind  den Use-Case-Diagrammen zu entnehmen   LINK


\section{Zielgruppe}
\label{sec:Einleitung:Zielgruppe}

Die primäre Zielgruppe von Neptune besteht aus Veranstaltern und Besuchern von Privatpartys, die zwischen 5-50 Besucher haben. Das Alter der Zielgruppe liegt dabei bei 18-35 Jahren, in dieser Altersgruppe ist von Vertrautheit bei der Bedingung von Smartphoneapps auszugehen. Darüber hinaus haben eine Mehrheit in dieser Altersgruppe einen Zugang zu Spotify, sowie ein aktuelles Smartphone. Dadurch sind die Zugangsbarrieren für die Benutzung von Neptune sehr niedrig.

Auch für Menschen außerhalb dieser primären Zielgruppe kann Neptune durch sein flexibles und einfaches Design interessant sein. Zum Beispiel ist mit der App auch möglich über die perfekte Entspannungsmusik beim Yoga abzustimmen oder das beliebteste Weihnachtslied in der Großfamilie zu ermitteln. 
Im Rahmen dieses PSE-Projekt wird die Zielgruppe auf Androidbenutzer eingeschränkt.

\chapter{Zielbestimmungen}
\label{chap:Zielbestimmungen}

\section{Musskriterien}
\label{sec:Zielbestimmungen:Musskriterien}
\begin{itemize}
    \item Mithilfe des Systems sollen User gemeinsam auf Veranstaltungen live über die während der Veranstaltung abgespielte Musik entscheiden können
    \item Hauptbestandteil des Systems ist zum einen ein Entscheidungssystem, welches Usern das Vorschlagen neuer Songs ermöglicht. Zum anderen ermöglicht das System den Usern, in verschiedenen Modi über die abzuspielenden Songs abzustimmen.
    \item Das System ist in der Lage, Usern einen durchsuchbaren Musikkatalog zur Verfügung zu stellen. Diese werden über entsprechende, auf dem Markt verfügbare Schnittstellen externer Anbieter bereitgestellt. Konkret wird dies im System durch die Einbindung mindestens einer externen API-Schnittstelle eines Musikstreaming-Anbieters realisiert.
    \begin{itemize}
        \item Das im Rahmen des Moduls "Praxis der Softwareentwicklung" zu entwickelnde System beschränkt sich hierbei zunächst auf einen konkreten Musikstreaming-Anbieter, namentlich auf das Angebot des schwedischen Unternehmens Spotify. Hierbei soll die Möglichkeit zur einfachen komplexen Ausweitung des Systems auf weitere Musikstreaming-Anbieter gegeben sein.
        \begin{itemize}
            \item Aufgrund von Einschränkungen seitens Spotify bezüglich der Entwicklerarbeit mit der von Spotify bereitgestellten API ist die vorgesehene Anzahl zulässiger User innerhalb einer gemeinsamen Session zunächst auf 25 User beschränkt.
            \item Aufgrund der Eigenschaften der Spotify-API ergibt sich darüber hinaus, dass für User ohne Spotify-Account sowie User ohne einen Spotify-Account mit einem Premium-Abonnement vorgesehen ist, dass diese nicht selbst neue Songs in der App vorschlagen können. Jedoch können diese User über von anderen Usern vorgeschlagene Songs abstimmen.
        \end{itemize}

        \end{itemize}
    \item Für eine Veranstaltung kann eine sogenannte Session durch den Host erstellt werden. Participants können dieser beitreten. Alle Entscheidungen über abzuspielende Musik finden innerhalb der Session statt.
    \item Das System verwendet zur Entscheidung über abzuspielende Songs ein nach diversen Kriterien konfigurierbares Abstimmungssystem.
    \begin{itemize}
        \item Zum einen sind verschiedene Abstimmungsmodi vorgesehen:
        \begin{itemize}
            \item (HINWEIS: Explizite Modi füge ich noch hinzu)
        \end{itemize}
        \item  (platzhalter zwecks Format)
    \end{itemize}
    \item Die User können genau eine der untenstehend aufgeführten Rollen bekleiden. Je nach der individuellen Ausgestaltung der Abstimmungsmodi ist die Rolle eines Users entscheidend in Hinblick auf die Entscheidungsfindung zur abzuspielenden Musik.
    \begin{itemize}
        \item Host: Der Host ist der Ersteller einer Session. Auf der einen Seite verwaltet er sie und kann sie auflösen, auf der anderen Seite wird die abzuspielende Musik über sein Endgerät abgespielt.
        \item Participant: Alle Teilnehmer einer Session, welche nicht der Host sind, bekleiden automatisch die Rolle Participant. Participants können basierend auf den gewählten Abstimmungsparametern neue Songs vorschlagen und/oder über vorgeschlagene Songs abstimmen. 
    \end{itemize}
    \item Die abzuspielende Musik wird über externe Anbieter abgespielt. Hierbei wird analog zur Einbindung von Musikstreaming-Anbietern bereitgestellter externer Schnittstellen vorgegangen.
    \item  Den Usern stehen für die Bedienung und das User-Interface der App folgende Sprachen zur Verfügung, zwischen welchen sie frei wechseln können:
    \begin{itemize}
        \item Englisch: Durch die weite Verbreitung der englischen Sprache sorgt ihre Unterstützung innerhalb der App dafür, dass die erreichbare Zielgruppe stark wächst, ohne direkt nativ eine Vielzahl von Sprachen unterstützen zu müssen.
        \item Deutsch: \textit{(--> muss hier eine Begruendung hin?)}
    \end{itemize}
    
    \end{itemize}


\section{Wunschkriterien}
\label{sec:Zielbestimmungen:Wunschkriterien}
\begin{itemize}
    \item In der App sollen Statistiken bezüglich des Ablaufs eines Events zur Verfügung stehen. (LASSEN, oder SAUBER)
    
    Statistik Ideen:
    Statistik wird beim Verlassen angezeigt und /oder durch einen button in der oberen Leiste
    Statistik als Bild ausgebbar und teilbar über Standartandroid Schnittstelle
    Statistik für alle Participant und Host einsehbar
    Mögliche Elemente der Statistik
    \begin{itemize}
        \item Song mit meisten Votes der Session
        \item Genre mit den meisten Votes der Session
        \item Artist mit den meisten Votes der Session
        \item Anzahl abgespielte Songs in der Session
        \item Dauer der Session
        \item Anzahl Teilnehmer in der Session (über GeräteId) ?
        \item Verteilte Herzen in Session
        \item meistgespielter Künstler / Genre
    \end{itemize}
  
    \item Im Falle dessen, dass die Teilnehmer keine neue Musik vorschlagen und die Warteschlange hierdurch leer ist, soll die App mittels einer Autoplay-Funktion dennoch kontinuierlich Musik abspielen. (soll das Kann-Kriterium bleiben ???)

\end{itemize}

\section{Abgrenzungskriterien}
\label{sec:Zielbestimmungen:Abgrenzungskriterien}
\begin{itemize}
    \item Musik soll nicht innerhalb der App selbst abgespielt werden.
    \item Im Rahmen des Projekts soll keine öffentlich zugängliche API und insgesamt keine Schnittstelle für weitere externe Anwendungen bereitgestellt werden
    \item Die App soll, abgesehen von einer Internetverbindung, keine weiteren alternativen Netzwerktechnologien wie Bluetooth o.ä. nutzen.

\end{itemize}



\chapter{Anforderungen an die Produktumgebung}
\label{chap:Produktumgebung}

\section{Softwareanforderungen}
\label{sec:Produktumgebung:Softwareanforderungen}

\begin{itemize}
    \item Spotify API
    \item ...
\end{itemize}

\section{Hardwareanforderungen}
\label{sec:Produktumgebung:Hardwareanforderungen}

\begin{itemize}
    \item Smartphone mit mindestens Android 7.0
    \item Server
\end{itemize}



\chapter{Produktdaten}
\label{chap:Produktdaten}

\section{Systemdaten}
\label{sec:Produktdaten:Systemdaten}

Als Systemdaten werden sämtliche Daten zur Generierung der Benutzeroberfläche gespeichert, sowie Sitzungs-IDs oder Geräte-Tokens und Daten zum Auffinden des Servers. (Evt. kommen noch weiter hinzu wie Benutzereinstellungen, Historiendaten (einfachere Suche vllt.),  App-Zustandsdaten, ...)

\section{Nutzerdaten}
\label{sec:Produktdaten:Nutzerdaten}

Falls das Account-System nicht umgesetzt wird, werden keine sensiblen Nutzerdaten verarbeitet und gespeichert. Ausschließlich die zum Aufruf der Musik-App API benötigten Daten (WELCHE SIND DAS ???) sowie Sitzungs-IDs oder Geräte-Tokens werden verarbeitet und gespeichert.



\chapter{Systemmodell}
\label{chap:Systemmodell}

Es erfolgt eine Teilung in Client und Server, wobei auf dem Server der größte Teil der Logik geschieht.

Für den Client könnte eine Model-View-Controller-Architektur verwendet werden. Besonders relevant ist dabei, dass Anfragen an die Musik-API sinnvoll eingegliedert werden, da diese einen großen Teil des App-Umfangs ausmachen. Vermutlich würde man sowohl Musik-API als auch Kommunkation mit dem Server in den Bereich des Controllers einordnen.

Hier evt. noch Grafiken einfügen...



\chapter{Benutzeroberfläche}
\label{chap:Benutzeroberfläche}

\section{Einführung}
\label{sec:Benutzeroberfläche:Einführung}
\textbf{} Die Benutzeroberfläche wird so gestaltet, dass diese einfach von Nutzern ohne Vorkenntnissen über die App genutzt werden kann. 

\section{Startmenü}
\label{sec:Benutzeroberfläche:Startmenü}
Der Nutzer landet beim Start der App immer im Startmenü.

\section{Participant Ansicht}
\label{sec:Benutzeroberfläche:participantAnsicht}
In der Participant Ansicht kann der Nutzer einer Gruppe beitreten und je nach Modus neue Lieder vorschlagen und liken.

\section{Host Ansicht}
\label{sec:Benutzeroberfläche:hostAnsicht}
In der Host Ansicht kann der Host einen Modus auswählen und Einstellungen für die Session definieren. Des weiteren kann er wie ein Participant Lieder hinzufügen und liken. Zusätzlich kann der Host Lieder aus der Queue löschen und Lieder aus der Vorschlagsliste zur Queue hinzufügen

\section{Grafikentwürfe}
\label{sec:Benutzeroberfläche:Grafikentwürfe}

\begin{figure}
   \begin{minipage}[b]{.4\linewidth} % [b] => Ausrichtung an \caption
      \includegraphics[width=5.5cm]{LATEX/Pflichtenheft/GraphicDesigns/startPage.png}
      \caption{startPage}
   \end{minipage}
   \hspace{2cm}% Abstand zwischen Bilder
   \begin{minipage}[b]{.4\linewidth} % [b] => Ausrichtung an \caption
      \includegraphics[width=5.5cm]{LATEX/Pflichtenheft/GraphicDesigns/userJoinGroupPage.png}
      \caption{userJoinGroupPage}
   \end{minipage}
   
   \begin{minipage}[b]{.4\linewidth} % [b] => Ausrichtung an \caption
      \includegraphics[width=5.5cm]{LATEX/Pflichtenheft/GraphicDesigns/userVotePage.png}
      \caption{userVotePage}
   \end{minipage}
   \hspace{2cm}% Abstand zwischen Bilder
   \begin{minipage}[b]{.4\linewidth} % [b] => Ausrichtung an \caption
      \includegraphics[width=5.5cm]{LATEX/Pflichtenheft/GraphicDesigns/userSearchPage.png}
      \caption{userSearchPage}
   \end{minipage}
\end{figure}

\begin{figure}
   \begin{minipage}[b]{.4\linewidth} % [b] => Ausrichtung an \caption
      \includegraphics[width=5.5cm]{LATEX/Pflichtenheft/GraphicDesigns/hostModusSelectPage.png}
      \caption{hostModusSelectPage}
   \end{minipage}
   \hspace{2cm}% Abstand zwischen Bilder
   \begin{minipage}[b]{.4\linewidth} % [b] => Ausrichtung an \caption
      \includegraphics[width=5.5cm]{LATEX/Pflichtenheft/GraphicDesigns/hostModusSettingsPage.png}
      \caption{hostModusSettingsPage}
   \end{minipage}
   
   \begin{minipage}[b]{.4\linewidth} % [b] => Ausrichtung an \caption
      \includegraphics[width=5.5cm]{LATEX/Pflichtenheft/GraphicDesigns/hostControlPage.png}
      \caption{hostControlPage}
   \end{minipage}
   \hspace{2cm}% Abstand zwischen Bilder
   \begin{minipage}[b]{.4\linewidth} % [b] => Ausrichtung an \caption
      \includegraphics[width=5.5cm]{LATEX/Pflichtenheft/GraphicDesigns/hostSearchPage.png}
      \caption{hostSearchPage}
   \end{minipage}
\end{figure}

\begin{figure}
   \begin{minipage}[b]{.4\linewidth} % [b] => Ausrichtung an \caption
      \includegraphics[width=5.5cm]{LATEX/Pflichtenheft/GraphicDesigns/shareLinkPopUpPage.png}
      \caption{shareLinkPopUpPage}
   \end{minipage}
   \hspace{2cm}% Abstand zwischen Bilder
   \begin{minipage}[b]{.4\linewidth} % [b] => Ausrichtung an \caption
      \includegraphics[width=5.5cm]{LATEX/Pflichtenheft/GraphicDesigns/userLeaveGroupPopupPage.png}
      \caption{userLeaveGroupPopupPage}
   \end{minipage}
   
   \begin{minipage}[b]{.4\linewidth} % [b] => Ausrichtung an \caption
      \includegraphics[width=5.5cm]{LATEX/Pflichtenheft/GraphicDesigns/hostDeleteGroupPopupPage.png}
      \caption{hostDeleteGroupPopupPage}
   \end{minipage}
\end{figure}

Spare Text:

Um ein übersichtliches Startmenü zu gewährleisten kann der Nutzer nur zwischen den Optionen "Gruppe beitreten" und "Gruppe erstellen" wählen. 

Im Startmenü kann der Nutzer entweder über die Buttons in der Mitte einer Gruppe beitreten, oder eine Gruppe erstellen. Über den Zurück-Knopf oben links verlässt der Nutzer die App.

Auf der Nutzer Abstimmungsseite sieht der Nutzer lieder bereits von Gruppenmitglieder vorgeschlagen und geliked wurden. Zudem kann er Lieder liken, welche er noch nicht geliked hat. Durch klicken des Lieder suchen Buttons gelangt der Nutzer zur Liedersuche. Über den Zurück-Knopf oben links gelangt der Nutzer zurück zum Startmenü. Dieser Vorgang muss über ein Pop-Up Fenster bestätigt werden, da gleichseitig die Gruppe verlassen wird.




\chapter{Qualitätszielbestimmungen}
\label{chap:Qualitätszielbestimmungen}

\textbf{Korrekte Funktionalität}: Die korrekte Funktionalität der App muss gewährleistet sein. Maßgeblich zur Definition dieser korrekten Funktionalität ist dieses Pflichtenheft und insbesondere die Musskriterien aus dem Kapitel der Zielbestimmungen (Kapitel \ref{sec:Zielbestimmungen:Musskriterien}).

\textbf{Benutzerfreundlichkeit}: Die App soll eine einfache und intuitive Benutzererfahrung bieten. Die Navigation innerhalb der App soll für alle Benutzer leicht verständlich sein, ohne dass diese eine Einführung in die App-Bedienung benötigen. Das maßgebliche Kriterium dafür ist, dass Benutzer der Zielgruppe (Kapitel \ref{sec:Einleitung:Zielgruppe}) die App bei der ersten Benutzung innerhalb von zwei Minuten problemlos navigieren können laut eigener Einschätzung. (ist so ok ???)

\textbf{Schnelligkeit}: Die App soll durch Reaktionsschnelligkeit ein flüssiges und ansprechendes Nutzungserlebnis ermöglichen. Das maßgebliche Kriterium dafür ist, dass jede Schaltflächenbedienung ein unmittelbares Feedback für den Benutzer auslöst, ohne dass dieser eine Wartezeit bemerkt, solange das Endgerät hinreichend aktuell ist. Hinreichend aktuell sind dabei insbesondere Geräte, die weniger als ein Jahr alt sind und die aktuellste verfügbare Betriebssystemversion installiert haben.

\textbf{Sicherheit der Benutzerdaten}: Es ist essentiell, dass die persönlichen Daten der Benutzer geschützt sind. Jegliche Interaktion mit und Datenverarbeitung durch die App muss datenschutzkonform sein und Datensicherheit bieten. Maßgeblich ist dafür die gesetzliche Regelung.

\textbf{Stabilität und Zuverlässigkeit}: Die App muss robust sein und selbst unter Last und bei mäßig vielen gleichzeitigen Benutzern stabil bleiben, sodass Abstürze oder unerwartete Fehler weitestgehend vermieden werden. Das maßgebliche Kriterium dafür sind die Testszenarien mit zahlreichen gleichzeitigen Usern. (solche Tests auch einfügen !!!)

\textbf{Wartbarkeit und Portierungsmöglichkeiten}: Die Architektur der App soll gut wartbar sein, um zukünftige Anpassungen oder Erweiterungen zu ermöglichen. Zudem soll die Architektur die Möglichkeit zur verhältnismäßig einfachen Portierung der App auf andere mobile Betriebssysteme wie iOS bieten. Dazu soll möglichst viel Logik auf dem Server geschehen, wie im Systemmodell beschrieben (Kapitel \ref{chap:Systemmodell}). Für diese Qualitätszielbestimmung existiert absichtlich kein gerichtsfestes Kriterium zur Überprüfung, maßgeblich ist die subjektive Einschätzung von Betrachtern des Systems. Die Erfüllung dieser Qualitätszielbestimmung ist im Verhältnis zu den anderen Bestimmungen nachrangig.

\textbf{Grundsätzliches zur Qualität}: Diese Qualitätszielbestimmungen werden während allen Projektphasen beachtet und in der Qualitätssicherungsphase intern abgenommen. Die Qualität der App besitzt während allen Projektphasen einen sehr hohen Stellenwert: Sie wird stets als Priorität betrachtet und in der Qualitätssicherungsphase abschließend und eingehend geprüft.



\chapter{Anwendungsfälle}
\label{chap:Anwendungsfälle}

In diesem Kapitel werden einige Anwendungsfälle der App mithilfe von Use Case Diagrammen anschaulich gemacht.

\section{App Start}
\label{sec:Anwendungsfälle:App Start}

\begin{figure}[h]
    \includegraphics[width = 18cm]{LATEX/Pflichtenheft/GraphicDesigns/Use Case App Start.png}
    \caption{App Start}
    \label{fig:Use Case App Start}
\end{figure}

\newpage

\section{General Mode}
\label{sec:Anwendungsfälle:General Mode}

\begin{figure}[h]
    \includegraphics[width = 18cm]{LATEX/Pflichtenheft/GraphicDesigns/Use Case General Mode.png}
    \caption{General Mode}
    \label{fig:Use Case General Mode}
\end{figure}

\newpage

\section{Artist Mode}
\label{sec:Anwendungsfälle:Artist Mode}

\begin{figure}[h]
    \includegraphics[width = 18cm]{LATEX/Pflichtenheft/GraphicDesigns/Use Case Artist Mode.png}
    \caption{Artist Mode}
    \label{fig:Use Case Artist Mode}
\end{figure}

\newpage

\section{Genre Mode}
\label{sec:Anwendungsfälle:Genre Mode}

\begin{figure}[h]
    \includegraphics[width = 18cm]{LATEX/Pflichtenheft/GraphicDesigns/Use Case Genre Mode.png}
    \caption{Genre Mode}
    \label{fig:Use Case Genre Mode}
\end{figure}

\newpage

\section{Playlist Mode}
\label{sec:Anwendungsfälle:Playlist Mode}

\begin{figure}[h]
    \includegraphics[width = 18cm]{LATEX/Pflichtenheft/GraphicDesigns/Use Case Playlist Mode.png}
    \caption{Playlist Mode}
    \label{fig:Use Case Playlist Mode}
\end{figure}



\chapter{Testfälle und -szenarien}
\label{chap:Tests}

In diesem Kapitel werden alle Testfälle und -szenarien definiert, die durch einen oder mehrere Benutzer des Produkts durchgeführt werden können. Die hier definierten Testfälle sollen deshalb explizit keine technischen Funktionalitäten testen, die ausreichend detailliert erst während der Entwurfsphase festgelegt werden. Solche technischen Funktionalitäten werden durch Unittests abgedeckt, die während der Entwurfs- und Implementierungsphase entworfen und implementiert werden.

\section{Testfälle}
\label{sec:Tests:Testfälle}

Testfälle sind Tests zu aus Benutzersicht atomaren Vorgängen. Jeder Testfall bezieht sich also auf eine einzelne atomare Benutzereingabe. Eine solche Benutzereingabe besteht entweder aus einem einzelnen Touch-Eingabe oder einer inhaltlich sehr stark zusammenhängenden Folge von Touch-Eingaben, die als atomar betrachtet wird.

\begin{enumerate}
    \item Starten und Laden der App
    \item Verlassen der App
    \item Beenden der App
    \item Beginnen des Sessionbeitrittsvorgangs
    \item Verlassen des Sessionbeitrittsvorgangs
    \item Eingabe des Zugangscodes zum Sessionbeitritt
    \item Bestätigung des Zugangscodes zum Sessionbeitritt
    \item Verlassen der Session (als Participant)
    \item Upvote für einen Song (als Participant)
    \item Nach geupvoteten Songs filtern (als Participant) (was heißt das ???)
    \item Entfernen von Upvote von einem Song (als Participant)
    \item Öffnen der Songsuche (als Active Participant)
    \item Verlassen der Songsuche (als Active Participant)
    \item Suchen eines Songs (als Active Participant)
    \item Vorschlagen eines Songs (als Active Participant)
    \item Beginnen der Sessionerstellung
    \item Verlassen der Sessionerstellung (in der Modus-Wahl)
    \item Wahl des Modus
    \item Zurückgehen in der Sessionerstellung (von den Modus-Details zur Modus-Wahl)
    \item Auswählen der Details des Modus
    \begin{itemize}
        \item Auswählen der Artists / der Genres /der Playlist (je nach Modus)
        \item Einstellen des Cool-Downs, wann ein Lied erneut vorgeschlagen werden kann
    \end{itemize}
    \item Bestätigung der Details des Modus
    \item Löschen der Gruppe (als Host)
    \item Editieren der Queue (muss noch detaillierter werden) (als Host)
    \item Editieren der Vorschlagsliste (muss noch detaillierter werden) (als Host)
    \item Sperren eines Liedes (muss noch detaillierter werden)
\end{enumerate}

\section{Grundlegende Testszenarien}
\label{sec:Tests:GrundlegendeTestszenarien}

Grundlegende Testszenarien sind eine überschaubare und besonders essentielle Folge von atomaren Testfällen, die eine Benutzerinteraktion durchspielen. Sie setzen sich aus Testfällen oder bereits zuvor definierten grundlegenden Testszenarien zusammen.

\subsection{Grundlegendes Testszenario 1 (G1): Erstellen einer Session mit Modus $M$}
\label{subsec:Tests:GrundlegendeTestszenarien:G1}
Ein User erstellt eine Session und wird so zu ihrem Host.
\begin{itemize}
    \item Starten und Laden der App
    \item Beginnen der Sessionerstellung
    \item Wahl des Modus $M$
    \item Auswählen der Details des Modus
    \item Bestätigung der Details des Modus
\end{itemize}


\subsection{Grundlegendes Testszenario 2 (G2): Beitritt einer Session mit Modus $M$ als Active Participant}
\label{subsec:Tests:GrundlegendeTestszenarien:G2}
Ein User, tritt als Active Participant einer Session mit Modus $M$ bei, die von einem anderen User als Host erstellt wurde.
Dazu muss bereits G1 (\ref{subsec:Tests:GrundlegendeTestszenarien:G1}) abgelaufen sein.
\begin{itemize}
    \item Starten und Laden der App
    \item Beginnen des Sessionbeitrittsvorgangs
    \item Eingabe des Zugangscodes zum Sessionbeitritt
    \item Bestätigung des Zugangscodes zum Sessionbeitritt
\end{itemize}


\subsection{Grundlegendes Testszenario 3 (G3): Beitritt einer Session mit Modus $M$ als Passive Participant}
\label{subsec:Tests:GrundlegendeTestszenarien:G3}
Ein User, tritt als Passive Participant einer Session mit Modus $M$ bei, die von einem anderen User als Host erstellt wurde.
Dazu muss bereits G1 (\ref{subsec:Tests:GrundlegendeTestszenarien:G1}) abgelaufen sein.
\begin{itemize}
    \item Starten und Laden der App
    \item Beginnen des Sessionbeitrittsvorgangs
    \item Eingabe des Zugangscodes zum Sessionbeitritt
    \item Bestätigung des Zugangscodes zum Sessionbeitritt
\end{itemize}

\subsection{Grundlegendes Testszenario 4 (G4): Vorschlagen eines Songs}
\label{subsec:Tests:GrundlegendeTestszenarien:G4}
Ein User, der schon als Active Particpant einer Session mit Modus $M$ beigetreten ist, die von einem anderen User als Host erstellt wurde, schlägt einen Song vor. \\
Dazu muss bereits G2 (\ref{subsec:Tests:GrundlegendeTestszenarien:G2}) abgelaufen sein.
\begin{itemize}
    \item Öffnen der Songsuche
    \item Suchen eines Songs
    \item Vorschlagen eines Songs
\end{itemize}

\subsection{Grundlegendes Testszenario 5 (G5): Upvoten eines Songs}
\label{subsec:Tests:GrundlegendeTestszenarien:G5}
Ein User, der schon als Participant einer Session mit Modus $M$ beigetreten ist, die von einem anderen User als Host erstellt wurde, votet einen Song up. \\
Dazu muss bereits G2 oder G3 (\ref{subsec:Tests:GrundlegendeTestszenarien:G2} oder \ref{subsec:Tests:GrundlegendeTestszenarien:G3}) abgelaufen sein.
\begin{itemize}
    \item Upvote für einen Song
\end{itemize}

\subsection{Grundlegendes Testszenario 6 (G6): Hinzufügen eines vorgeschlagenen Songs zur Queue}
\label{subsec:Tests:GrundlegendeTestszenarien:G6}
Der Host fügt einen von Active Participants vorgeschlagenen Song in einer Session mit Modus $M$ zur Queue hinzu.
Dazu muss bereits G2 (\ref{subsec:Tests:GrundlegendeTestszenarien:G2}) und danach G4 (\ref{subsec:Tests:GrundlegendeTestszenarien:G4}) abgelaufen sein.
\begin{itemize}
    \item Editieren der Queue (muss noch detaillierter werden) (als Host) (?????)
\end{itemize}



\section{Erweiterte Testszenarien}
\label{sec:Tests:ErweiterteTestszenarien}

Erweiterte Testszenarien sind eine nicht-essentielle und möglicherweise längere Folge von atomaren Testfällen, die eine Benutzerinteraktion durchspielen. Sie setzen sich aus Testfällen und grundlegenden Testszenarien zusammen.

\subsection{Erweitertes Testszenario 1 (E1): Ein vorgeschlagener und einmal upgevoteter Song wird der Queue hinzugefügt}
\label{subsec:Tests:ErweiterteTestszenarien:E1}
Ein Host (H) erstellt eine Gruppe, der ein Active Participant (AP) und ein Passive Partipant (PP) beitreten. AP schlägt einen Song vor, den PP upvotet und H dann in die Queue hinzufügt.
\begin{itemize}
    \item G1 (H)
    \item G2 (AP)
    \item G3 (PP)
    \item G4 (AP)
    \item G5 (PP)
    \item G6 (H)
\end{itemize}



\chapter{Entwicklungsumgebung}
\label{chap:Entwicklungsumgebung}

\section{Software}
\label{sec:Entwicklungsumgebung:Software}

\begin{itemize}
    \item Entwicklung
    \begin{itemize}
        \item Android Studio Giraffe (2022.3.1)
    \end{itemize}
    
    \item Versionsverwaltung
    \begin{itemize}
        \item GitHub
    \end{itemize}

    \item Modellierung
    \begin{itemize}
        \item Creately
    \end{itemize}

    \item Sonstige Software
    \begin{itemize}
        \item LATEX (Dokumentation)
        \item ...
    \end{itemize}
    
\end{itemize}

\section{Hardware}
\label{sec:Entwicklungsumgebung:Hardware}

\begin{itemize}
    \item Diverse Standard PCs
    \item Diverse Android Smartphones mit mindestens Android 7.0
\end{itemize}


\chapter{Begriffserklärungen}
\label{chap:Begriffserklärungen}

\textbf{System}
 - Das gesamte zu entwickelnde Softwaresystem, Überbegriff von App und Server.

\textbf{App}
 - Der Teil des Softwaresystems, der auf dem Android-Gerät des Nutzers läuft.

\textbf{Server}
 - Der Teil des Softwaresystems, der nicht auf dem Android-Gerät des Nutzers, sondern auf einem zentralen und externen Server läuft.

\textbf{Produkt}
 - Der Zustand, in dem das System nach Fertigstellung sein wird.

\textbf{Session}
 - Eine Gruppe aus Participants und einem Gruppenersteller, dem Host. In ihr kann Musik vorgeschlagen und vom Host abgespielt werden.

\textbf{Host}
 - Ersteller einer Session, der (vorgeschlagene) Musik abspielt (kein Participant).

\textbf{Participant}
 - Teilnehmer einer Session (nicht der Host). Überbegriff für Active Participant und Passive Participant.

\textbf{Active Participant}
 - Mit dem Musikdienst verknüpfter Participant, der Musik vorschlagen und darüber abstimmen kann.

\textbf{Passive Participant}
 - Nicht mit dem Musikdienst verknüpfter Participant, der nur über Vorschläge von anderen abstimmen kann.

\textbf{User}
 - Jeder Benutzer der App, Participant und Host sind User.

\textbf{Musikdienst}
 - Der Musik-Streaminganbieter, über den die App Musik abspielt. Bei unserer App ist das Spotify.
 
\textbf{Queue}
 - Liste von Liedern, welche als nächstes durch den Musikdienst (Spotify) abgespielt werden.

\textbf{Vorschlagsliste}
 - Liste von Liedern, die von einem User vorgeschlagen wurden.


\end{document}
